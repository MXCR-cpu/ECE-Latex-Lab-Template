\usepackage[american]{babel}
\usepackage{graphicx}
\usepackage{amsmath}
\usepackage{amssymb}
\usepackage{float}
\usepackage{tikz}
\usepackage{circuitikz}
\usepackage{pgfplots}
\usepackage{pgfplotstable}
\usepackage{siunitx}
\usepackage{booktabs}
\usepackage{listings}
\usepackage{subcaption}
\usepackage{minted}
\usepackage{calc}
\usepackage{mdframed}
\usepackage{csquotes} % One of the things you learn about LaTeX is at some level, it's like magic. The references weren't printing as they should without this line, and the guy who wrote the package included it, so here it is. Because LaTeX reasons.
\usepackage[style=apa,sortcites=true,sorting=nyt,backend=biber]{biblatex}
% biblatex: loads the package that will handle the bibliographic info. Other option is natbib, which allows for more customization
% - style=apa: sets the reference format to use apa (albeit the 6th edition)
\usepackage[T1]{fontenc} 
\usepackage{xstring} % Alternately, you can comment out or delete these two commands and just use the Overleaf default font. So many choices!
\usepackage{lmodern}

\usetikzlibrary{automata}
\usetikzlibrary{graphs}
\usetikzlibrary{graphdrawing}
\usetikzlibrary{graphs.standard}
\usetikzlibrary{perspective}
\usetikzlibrary{positioning}

\usegdlibrary{circular}
\usegdlibrary{force}
\usegdlibrary{layered}
